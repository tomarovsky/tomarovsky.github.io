%-------------------------
% Resume in Latex
% Author : Jake Gutierrez
% Based off of: https://github.com/sb2nov/resume
% License : MIT
%------------------------

\documentclass[letterpaper,11pt]{article}

\usepackage{latexsym}
\usepackage[empty]{fullpage}
\usepackage{titlesec}
\usepackage{marvosym}
\usepackage[usenames,dvipsnames]{color}
\usepackage{verbatim}
\usepackage{enumitem}
\usepackage[hidelinks,colorlinks=true,urlcolor=black]{hyperref}
\usepackage{fancyhdr}
\usepackage[english]{babel}
\usepackage{tabularx}
\input{glyphtounicode}


%----------FONT OPTIONS----------
% sans-serif
% \usepackage[sfdefault]{FiraSans}
% \usepackage[sfdefault]{roboto}
% \usepackage[sfdefault]{noto-sans}
% \usepackage[default]{sourcesanspro}

% serif
% \usepackage{CormorantGaramond}  % Optional
\usepackage{charter}


\pagestyle{fancy}
\fancyhf{} % clear all header and footer fields
\fancyfoot{}
\renewcommand{\headrulewidth}{0pt}
\renewcommand{\footrulewidth}{0pt}

% Adjust margins
\addtolength{\oddsidemargin}{-0.5in}
\addtolength{\evensidemargin}{-0.5in}
\addtolength{\textwidth}{1in}
\addtolength{\topmargin}{-.5in}
\addtolength{\textheight}{1.0in}

\urlstyle{same}

\raggedbottom
\raggedright
\setlength{\tabcolsep}{0in}

% \definecolor{mycolor}{RGB}{0, 128, 0} % Define a custom color (RGB model)
\definecolor{mycolor}{RGB}{49, 89, 152} % GPT
\definecolor{urlcolor}{RGB}{44, 79, 152} % GPT
% \definecolor{mycolor}{RGB}{57, 114, 180}    % Zuobai and Runzhe
% \definecolor{mycolor}{RGB}{37, 150, 190} % Define a custom color (RGB model)
% \definecolor{itemcolor}{RGB}{37, 150, 190}
\definecolor{itemcolor}{RGB}{0, 0, 0}
% Sections formatting
\titleformat{\section}{
  \vspace{-4pt}\color{mycolor}\scshape\raggedright\large
}{}{0em}{}[\color{mycolor}\titlerule \vspace{-5pt}]

% Ensure that generate pdf is machine readable/ATS parsable
\pdfgentounicode=1

%-------------------------
% Custom commands
\newcommand{\resumeItem}[1]{
  \item\small{
    {#1 \vspace{-2pt}}
  }
}

\newcommand{\resumeSubheading}[4]{
  \vspace{-2pt}\item
    \begin{tabular*}{0.97\textwidth}[t]{l@{\extracolsep{\fill}}r}
      \textbf{#1} & #2 \\
      \textit{\small#3} & \textit{\small #4} \\
    \end{tabular*}\vspace{-5pt}
}

\newcommand{\resumeSubSubheading}[2]{
    \item
    \begin{tabular*}{0.97\textwidth}{l@{\extracolsep{\fill}}r}
      \textit{\small#1} & \textit{\small #2} \\
    \end{tabular*}\vspace{-7pt}
}

\newcommand{\resumeProjectHeading}[2]{
    \item
    \begin{tabular*}{0.97\textwidth}{l@{\extracolsep{\fill}}r}
      #1 & #2 \\
    \end{tabular*}\vspace{-7pt}
}


\newcommand{\resumeSubItem}[1]{\resumeItem{#1}\vspace{-4pt}}

\renewcommand\labelitemii{$\vcenter{\hbox{\tiny$\bullet$}}$}

\newcommand{\resumeSubHeadingListStart}{\begin{itemize}[leftmargin=0.15in, label={}]}
\newcommand{\resumeSubHeadingListEnd}{\end{itemize}}
\newcommand{\resumeItemListStart}{\begin{itemize}[label={\color{itemcolor}•}]}
\newcommand{\resumeItemListEnd}{\end{itemize}\vspace{-5pt}}

% For Experience
\newcommand{\resumeExpListStart}{\begin{itemize}[leftmargin=0.15in, label={\color{itemcolor}$\circ$}, itemsep=1\itemsep]}
\newcommand{\resumeExpListEnd}{\end{itemize}}
\newcommand{\resumeSubExpListStart}{\begin{itemize}[leftmargin=\leftmargin, label={\color{itemcolor}•}, itemsep=2\itemsep]}
\newcommand{\resumeSubExpListEnd}{\end{itemize}}

\newcommand{\resumeExpItem}[1]{
  \item{
    {\textbf{#1} \vspace{-1pt}}
  }
}

\newcommand{\resumeExpBriefItem}[1]{
  \item{
    {{\small#1} \vspace{-1pt}}
  }
}


\newcommand{\resumeSubExpItem}[1]{
  \item\small{
    {#1 \vspace{-2pt}}
  }
}

\newcommand{\resumeText}{\vspace{3pt}}
\newcommand{\resumePosition}[1]{
    \item
    \begin{tabular*}{0.97\textwidth}{l}
      \small#1 \\
    \end{tabular*}\vspace{-7pt}
}
\newcommand{\resumeDataItem}[2]{
  \item\small{
    #1 \hfill #2 \vspace{-2pt}
  }
}
\newcommand{\resumePublication}[2]{
  \vspace{-2pt}\item
    \begin{tabular*}{0.97\textwidth}[t]{p{0.9\textwidth}@{\extracolsep{\fill}}r}
      \textbf{#1} & \\
      \textit{\small#2} & \\
    \end{tabular*}\vspace{-2pt}
}

% \newcommand{\resumeHonor}[2]{
%   \vspace{-2pt}\item
%     \begin{tabular*}{0.97\textwidth}[t]{l@{\extracolsep{\fill}}r}
%       \textbf{#1} & #2 \\
%     \end{tabular*}\vspace{-7pt}
% }
\newcommand{\resumeHonor}[2]{
  \vspace{-2pt}\item
    {\small#1} \hfill {\small#2}
  \vspace{-7pt}
}

%-------------------------------------------
%%%%%%  RESUME STARTS HERE  %%%%%%%%%%%%%%%%%%%%%%%%%%%%


\begin{document}

%----------HEADING----------
% \begin{tabular*}{\textwidth}{l@{\extracolsep{\fill}}r}
%   \textbf{\href{http://sourabhbajaj.com/}{\Large Sourabh Bajaj}} & Email : \href{mailto:sourabh@sourabhbajaj.com}{sourabh@sourabhbajaj.com}\\
%   \href{http://sourabhbajaj.com/}{http://www.sourabhbajaj.com} & Mobile : +1-123-456-7890 \\
% \end{tabular*}

\begin{center}
    \textbf{\Huge \scshape \color{mycolor}Yangtian Zhang} \\ \vspace{10pt}
    % \small 123-456-7890 $|$ 
    \href{mailto:yangtian.zhang@yale.edu}{Email: \underline{yangtian.zhang@yale.edu}} $|$ 
    \href{https://zytzrh.github.io/}{Homepage: \underline{https://zytzrh.github.io/}} $|$
    \href{https://github.com/zytzrh}{Github: \underline{https://github.com/zytzrh}}\\ \vspace{3pt}
\end{center}

\hypersetup{urlcolor=urlcolor}
% \begin{center}
%     \textbf{\Huge \scshape Yangtian Zhang} \\ \vspace{1pt}
%     % \small 123-456-7890 $|$ 
%     \href{mailto:zytzrh@gmail.com}{Email: zytzrh@gmail.com} \\
%     \href{https://zytzrh.github.io/}{Homepage: \underline{https://zytzrh.github.io/}} \\
%     \href{https://github.com/zytzrh}{Github: \underline{https://github.com/zytzrh}}
% \end{center}

%-----------Interest-----------
\section{Research Interest}
\resumeText{My primary research interests lie within the domain of Generative AI and Multi-Modal Foundation Models. My goal is to develop generative approaches for solving diverse real-world issues, including those in vision, language, and scientific domains.}

%-----------EDUCATION-----------
\section{Education}
  \resumeSubHeadingListStart
    \resumeSubheading
      {Shanghai Jiao Tong University}{Shanghai, China}
      {Bachelor of Engineering (B.Eng.) in Computer Science, \textit{Summa Cum Laude}}{Sept. 2018 -- June 2022}
      \resumeItemListStart
        \resumeItem{Member of \textbf{ACM Honors Class}, which is an elite CS program for top 5\% talented students.}
        \resumeItem{GPA: \textbf{90.79 / 100}
        % Ranking: \textbf{5 / 31}
        }
      \resumeItemListEnd
    \resumeSubheading
      {Yale University}{New Haven, US}
      {Doctor of Philosophy (Ph.D.) student in Computer Science}{Sept. 2024 -- }
      \resumeItemListStart
        \resumeItem{Co-advised by \textbf{Prof. Rex Ying} and \textbf{Prof. David van Dijk}.}
        \resumeItem{Working on multi-modal foundation models and generative modeling techniques.}
      \resumeItemListEnd
  \resumeSubHeadingListEnd

% %-----------Professional Services-----------
% \section{Professional Services}
%  \begin{itemize}[leftmargin=0.15in, label={}]
%     \small{\item{
%      {Reviewer, }\textbf{Neural Information Processing Systems (NeurIPS)}{, 2023} \\
%      {Reviewer, }\textbf{International Conference on Machine Learning (ICML)}{, 2023} \\
%      {Reviewer, }\textbf{International Conference on Learning Representations (ICLR)}{, 2024} \\
%     }}
%  \end{itemize}

% %-----------Teaching Experience-----------
% \section{Teaching Experience}
% % During my undergraduate studies, I serve as a teaching assistant, and I was deeply involved in the course design of the following class: 
% % \vspace{-3pt}
%  \begin{itemize}[leftmargin=0.15in, label={}]
%     \small{\item{
%      {Teaching Assistant: }{CS158: Data Structures and Algorithms}{, 2020} \\
%      {Teaching Assistant: }{MS125: Principle and Practice of Computer Algorithms}{, 2020} \\
%      {Teaching Assistant: }{CS420: Machine Learning}{, 2021} \\
%     }}
%  \end{itemize}

%%-----------Publications-----------
\section{Publications}
\resumeSubHeadingListStart
    \resumePublication{Non-Markovian Discrete Diffusion with Causal Language Models}
    {\textbf{Yangtian Zhang*}, Sizhuang He*, Daniel Levine, Lawrence Zhao, David Zhang, Syed A Rizvi, Emanuele Zappala, Rex Ying, David van Dijk (NeurIPS 2025)}

    \resumePublication{Flow Matching for Collaborative Filtering}
    {Chengkai Liu*, \textbf{Yangtian Zhang*}, Jianling Wang, Rex Ying, James Caverlee (KDD 2025)}

    \resumePublication{CaLMFlow: Volterra Flow Matching using Causal Language Models}
    {Sizhuang He, Daniel Levine, Ivan Vrkic, Marco Francesco Bressana, David Zhang, Syed Asad Rizvi, \textbf{Yangtian Zhang}, Emanuele Zappala, David van Dijk }
    
    \resumePublication{DiffPack: A Torsional Diffusion Model for Autoregressive Protein Side-Chain Packing}
    {\textbf{Yangtian Zhang*}, Zuobai Zhang*, Bozitao Zhong, Sanchit Misra, Jian Tang (NeurIPS 2023)}
    
    \resumePublication{E3Bind: An End-to-End Equivariant Network for Protein-Ligand Docking}
    {\textbf{Yangtian Zhang*}, Huiyu Cai*, Chence Shi, Bozitao Zhong, Jian Tang (ICLR 2023)}



    \resumePublication{PEER: A Comprehensive and Multi-Task Benchmark for Protein Sequence Understanding}
    {Minghao Xu, Zuobai Zhang, Jiarui Lu, Zhaocheng Zhu, \textbf{Yangtian Zhang}, Ma Chang, Runcheng Liu, Jian Tang (NeurIPS 2022 Datasets and Benchmarks Track)}

    \resumePublication{Torchdrug: A powerful and flexible machine learning platform for drug discovery}
    {Zhaocheng Zhu, Chence Shi, Zuobai Zhang, Shengchao Liu, Minghao Xu, Xinyu Yuan, \textbf{Yangtian Zhang}, Junkun Chen, Huiyu Cai, Jiarui Lu, Chang Ma, Runcheng Liu, Louis-Pascal Xhonneux, Meng Qu, Jian Tang }


    \resumePublication{A Survey on Diffusion Models for Recommender Systems}
    {Jianghao Lin, Jiaqi Liu, Jiachen Zhu, Yunjia Xi, Chengkai Liu, \textbf{Yangtian Zhang}, Yong Yu, Weinan Zhang (CoRR 2024)}



\resumeSubHeadingListEnd


%-----------EXPERIENCE-----------
\section{Research Experience}

  \resumeSubHeadingListStart
    \resumeSubheading
      {Google Deepmind}{April. 2024 -- July 2024}
      {Student Researcher, advised by \textbf{Dr. Jiaxin Shi} and \textbf{Dr. Michalis Titsias}}{New York, USA}
    \resumeExpListStart
        \resumeExpBriefItem{Conducted research on discrete generative modeling and discrete diffusion.}
    \resumeExpListEnd
    % MSRA
    \resumeSubheading
      {Microsoft Research AI4Science}{April. 2024 -- July 2024}
      {Research Intern, advised by \textbf{Dr. Tao Qin}}{Beijing, China}
    \resumeExpListStart
        \resumeExpBriefItem{Conducted research on multi-modal foundation model for AI-driven scentific discovery.}
    \resumeExpListEnd
  
  
      
      
    % Mila
    \resumeSubheading
      {Mila - Quebec AI Institute}{Sept. 2021 -- April 2024}
      {Graduate Research Assistant, advised by \textbf{Prof. Jian Tang}}{Montreal, Canada}
    \resumeExpListStart
    % \resumeExpItem{Research Topic: Generative AI, Geometric Deep Learning, Graph Representation Learning}
    \resumeExpBriefItem{Developed non-euclidean generative modelling with applications in protein science. See \href{https://arxiv.org/abs/2306.01794}{paper} here}
        % \resumeSubExpListStart
        %     \resumeSubExpItem{Analyzed the overparameterization and inefficiency issue of previous protein sidechain packing(PSCP) algorithm.}
        %     \resumeSubExpItem{Formulated a Riemannian diffusion process to better address the above problem in a generative manner. Developed DiffPack, an autoregressive diffusion model to predict the score in SE(3)-invariant torsion space.}
        %     \resumeSubExpItem{
        %     % Achieved improvement of 13.5\% in angle accuracy on CASP testset.
        %     Paper has been accepted to NeurIPS 2023. See \href{https://arxiv.org/abs/2306.01794}{paper} here.}
        % \resumeSubExpListEnd
    \resumeExpBriefItem{Developed geometric deep learning techniques to molecular docking. See \href{https://arxiv.org/abs/2210.06069}{paper} here.}
        % \resumeSubExpListStart
        %     \resumeSubExpItem{Reformulated the traditional molecule docking pipeline into an iterative refinement process (akin to AlphaFold2), enabling more rapid inference of protein-ligand conformations.}
        %     \resumeSubExpItem{Proposed E3Bind, an End-to-End E(3) equivariant network to update the atom coordinate in each iteration. The model comprises an expressive geometric-aware encoder and an equivariant coordinate update module.}
        %     \resumeSubExpItem{Paper has been accepted to ICLR 2023. See \href{https://arxiv.org/abs/2210.06069}{paper} here.}
        % \resumeSubExpListEnd

      % \resumeExpBriefItem{Foundation Model for Protein Sequence Understanding. See \href{https://arxiv.org/abs/2206.02096}{paper} here.}
        % \resumeSubExpListStart
        %     \resumeSubExpItem{Proposed a comprehensive multi-task learning benchmark to evaluate the performance of mainstream language architectures in biological sequence understanding (e.g. protein function prediction, protein-protein interaction) }
        %     \resumeSubExpItem{Paper has been accepted to NeurIPS 2022 Datasets and Benchmarks Track. See \href{https://arxiv.org/abs/2206.02096}{paper} here.}
        % \resumeSubExpListEnd
    \resumeExpListEnd
    
    % SJTU
    \resumeSubheading
      {APEX Lab, Shanghai Jiao Tong University}{Sept. 2020 -- June 2022}
      {Undergraduate Research Assistant, advised by \textbf{Prof. Weinan Zhang}}{Shanghai, China}
    \resumeExpListStart
    % \resumeExpItem{Research Topic: Reinforcement Learning, Imitation Learning}
    \resumeExpBriefItem{Designed a Multi-Step Generative Adversarial Imitation Learning framework. See \href{https://zytzrh.github.io/assets/pdf/AutoGAIL.pdf}{paper} here.}
        % \resumeSubExpListStart
        %     \resumeSubExpItem{Analyzed the roll-out discrepancy and the sample complexity of Multi Step Generative Adversarial Imitation Learning (GAIL).}
        %     \resumeSubExpItem{Proposed AutoGAIL, an auto curriculum version of T-step Occupancy Measure matching algorithm, to achieve better trade-off between rollout discrepancy and the sample complexity. See \href{https://zytzrh.github.io/assets/pdf/AutoGAIL.pdf}{paper draft} here.}
        % \resumeSubExpListEnd
    % \resumeExpBriefItem{Equivariant Imitation Learning}
        % \resumeSubExpListStart
        %     \resumeSubExpItem{Formulate the imitation learning framework in Symmetric Markov Decision Process.}
        %     \resumeSubExpItem{Proposed an equivariant imitation learning algorithm, EquiGAIL, which achieves better sample efficiency compared with vanila imitation learning algorithm in various RL environments (e.g. MolGym).}
        % \resumeSubExpListEnd
    \resumeExpListEnd
      


  \resumeSubHeadingListEnd

%-----------PROJECTS-----------
\section{Selected Projects}
    \resumeSubHeadingListStart
      \resumeProjectHeading
          {\textbf{TorchDrug: A Powerful and Flexible Machine Learning Platform for Drug Discovery}}{}
          \resumeItemListStart
            \resumeItem{Played a pivotal role in the development of TorchDrug, a robust ML platform for drug discovery, assisting in supporting 6 tasks and implementing over 25 models, specializing in bioinformatics applications.}
            \resumeItem{Over 1,300 stars and 40,000 downloads. See \href{https://torchdrug.ai/}{project} here.}
          \resumeItemListEnd
      \resumeProjectHeading
          {\textbf{TorchProtein: A Specialized Machine Learning Library for Protein Science}}{}
          \resumeItemListStart
            \resumeItem{Played a pivotal role in the development of TorchProtein, a specialized extension of TorchDrug, focusing on implementing representation learning models for both protein sequences and structures. See \href{https://torchprotein.ai/}{project} here.}
          \resumeItemListEnd
      \resumeProjectHeading
          {\textbf{Hands-on-RL: A Comprehensive Chinese Tutorial for Reinforcement Learning}}{}
          \resumeItemListStart
            \resumeItem{Contributed to the tutorial by authoring sections related to imitation learning, synthesizing complex RL concepts into accessible content, and implementing practical examples on Jupyter notebooks. See \href{https://github.com/boyu-ai/Hands-on-RL}{project} here.}
          \resumeItemListEnd
    \resumeSubHeadingListEnd

%%-----------Honors & Awards-----------
\section{Honors \& Awards}
    \resumeSubHeadingListStart
        \resumeHonor{\textbf{Irving T. Ho Memorial Scholarship} (4 undergrads per year)}{2021}
        \resumeHonor{\textbf{Zhiyuan Honorary Scholarship} (Top 5\% in SJTU)}{2018-2021}
        \resumeHonor{\textbf{Shanghai Jiao Tong University Excellent Scholarship} (Top 10\% in SJTU)}{2018-2021}
        \resumeHonor{\textbf{Outstanding Graduate Student}}{2022}
        \resumeHonor{\textbf{Meritorious Winner in Mathematical Contest in Modeling} (Top 7\%)}{2020}
        % \resumeHonor{\textbf{First Prize, National College Student Physics Competition}}{2020}
        \resumeHonor{\textbf{First Prize, National Olympiad in Physics}}{2017}
        
    \resumeSubHeadingListEnd

%%-----------Services-----------
\section{Services}
  \resumeSubHeadingListStart
        \resumePosition{\textbf{Academic Reviewer}}
            \resumeItemListStart
                \resumeDataItem{Neural Information Processing Systems (NeurIPS)}{2023-2024}
                \resumeDataItem{International Conference on Machine Learning (ICML)}{2023-2024}
                \resumeDataItem{International Conference on Learning Representations (ICLR)}{2024}
                \resumeDataItem{Association for the Advancement of Artificial Intelligence (AAAI)}{2024}
            \resumeItemListEnd
        \resumePosition{\textbf{Teaching Assistant} $|$ \emph{Shanghai Jiao Tong University}}
            \resumeItemListStart
                \resumeDataItem{CS158: Data Structures and Algorithms}{2020}
                \resumeDataItem{MS125: Principle and Practice of Computer Algorithms}{2020}
                \resumeDataItem{CS420: Machine Learning}{2021}
            \resumeItemListEnd
  \resumeSubHeadingListEnd
        








%
%-----------PROGRAMMING SKILLS-----------
\section{Technical Skills}
 \begin{itemize}[leftmargin=0.15in, label={}]
    \small{\item{
     \textbf{Programming Languages}{: Python, C/C++, Java, MATLAB, Verilog-HDL} \\
     \textbf{Deep Learning Packages: }{: PyTorch, TensorFlow, Keras} \\
     \textbf{Language}{: Mandarin Chinese(Native), English(Fluent)} \\
    }}
 \end{itemize}


%-------------------------------------------
\end{document}
